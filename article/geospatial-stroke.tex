\documentclass[]{article}
\usepackage{lmodern}
\usepackage{amssymb,amsmath}
\usepackage{ifxetex,ifluatex}
\usepackage{fixltx2e} % provides \textsubscript
\ifnum 0\ifxetex 1\fi\ifluatex 1\fi=0 % if pdftex
  \usepackage[T1]{fontenc}
  \usepackage[utf8]{inputenc}
\else % if luatex or xelatex
  \ifxetex
    \usepackage{mathspec}
  \else
    \usepackage{fontspec}
  \fi
  \defaultfontfeatures{Ligatures=TeX,Scale=MatchLowercase}
\fi
% use upquote if available, for straight quotes in verbatim environments
\IfFileExists{upquote.sty}{\usepackage{upquote}}{}
% use microtype if available
\IfFileExists{microtype.sty}{%
\usepackage{microtype}
\UseMicrotypeSet[protrusion]{basicmath} % disable protrusion for tt fonts
}{}
\usepackage[margin=1in]{geometry}
\usepackage{hyperref}
\hypersetup{unicode=true,
            pdftitle={A review of software tools, data and services for geospatial analysis of stroke services},
            pdfauthor={Nicholas Tierney, Mark Padgham, Michael Sumner, Geoff Boeing, Richard Beare},
            pdfborder={0 0 0},
            breaklinks=true}
\urlstyle{same}  % don't use monospace font for urls
\usepackage{graphicx,grffile}
\makeatletter
\def\maxwidth{\ifdim\Gin@nat@width>\linewidth\linewidth\else\Gin@nat@width\fi}
\def\maxheight{\ifdim\Gin@nat@height>\textheight\textheight\else\Gin@nat@height\fi}
\makeatother
% Scale images if necessary, so that they will not overflow the page
% margins by default, and it is still possible to overwrite the defaults
% using explicit options in \includegraphics[width, height, ...]{}
\setkeys{Gin}{width=\maxwidth,height=\maxheight,keepaspectratio}
\IfFileExists{parskip.sty}{%
\usepackage{parskip}
}{% else
\setlength{\parindent}{0pt}
\setlength{\parskip}{6pt plus 2pt minus 1pt}
}
\setlength{\emergencystretch}{3em}  % prevent overfull lines
\providecommand{\tightlist}{%
  \setlength{\itemsep}{0pt}\setlength{\parskip}{0pt}}
\setcounter{secnumdepth}{0}
% Redefines (sub)paragraphs to behave more like sections
\ifx\paragraph\undefined\else
\let\oldparagraph\paragraph
\renewcommand{\paragraph}[1]{\oldparagraph{#1}\mbox{}}
\fi
\ifx\subparagraph\undefined\else
\let\oldsubparagraph\subparagraph
\renewcommand{\subparagraph}[1]{\oldsubparagraph{#1}\mbox{}}
\fi

%%% Use protect on footnotes to avoid problems with footnotes in titles
\let\rmarkdownfootnote\footnote%
\def\footnote{\protect\rmarkdownfootnote}

%%% Change title format to be more compact
\usepackage{titling}

% Create subtitle command for use in maketitle
\newcommand{\subtitle}[1]{
  \posttitle{
    \begin{center}\large#1\end{center}
    }
}

\setlength{\droptitle}{-2em}

  \title{A review of software tools, data and services for geospatial analysis of
stroke services}
    \pretitle{\vspace{\droptitle}\centering\huge}
  \posttitle{\par}
    \author{Nicholas Tierney, Mark Padgham, Michael Sumner, Geoff Boeing, Richard
Beare}
    \preauthor{\centering\large\emph}
  \postauthor{\par}
    \date{}
    \predate{}\postdate{}
  

\begin{document}
\maketitle

\hypertarget{introduction}{%
\section{Introduction}\label{introduction}}

In this article we review a family of computational techniques and
services, collectively termed geospatial analysis tools, that can be
applied to a range of questions relevant to stroke services. Geospatial
analysis tools allow manipulation and modelling of geospatial data.
These tools, data, and modelling techniques have a long track record in
the quantitative geography, city and regional planning, and civil
engineering research literatures. Geospatial data, in the context of
stroke research, includes the location of patients and treatment
centres, routes through the road network linking patients to treatment
centres, geographic and administrative region boundaries (e.g.~post
codes, government areas, national boundaries) and disease incidence and
demographic information associated with such regions.

Recent advances in acute stroke therapy, in the form of endovascular
clot retrieval and clot busting drugs, are extremely effective when
treatment can be delivered within a relatively short time window
following stroke. There are many factors affecting the time delay
between stroke onset and treatment, one of which is transport of the
patient to a treatment center. Geospatial approaches have been used to
analyse the delivery of emergency clot retrieval services (Phan et al.
2017) {[}a{]} and to evaluate ``Drip and Ship'' approaches in specific
locales (Milne et al. 2017) {[}b{]} and population level access to
services (Adeoye et al. 2014) {[}c{]}.

Geospatial tools can be used to analyse and visualise geospatial data,
such as patient collection location as well as perform a range of
simulations at varying levels of detail. For example, in Phan et al
(Phan et al. 2017) {[}d{]}, travel times between a set of randomly
generated addresses and a set of possible destinations were estimated
using queries to several Google Application Programming Interfaces
(API{[}e{]} (Richard: ``Is this referring to the Google Distance
\emph{Matrix} API?'')), allowing various configuration of the treatment
network to be tested. Combination of the resulting catchment areas with
demographic data allowed loadings to be estimated.

The studies cited above were constructed using a series of standard
geospatial analysis components. In this article we will introduce these
components and provide examples of how they can be used to answer health
related questions. Examples are implemented using open source software,
specifically R and python, and source code provided so that readers can
reproduce and modify them (R Core Team 2018) (Sanner and others 1999)
{[}f{]}. Geospatial analysis tools have traditionally been the domain of
specialist commercial software and vendors, however this is no longer
the case, with a range of open source options available to researchers.
These tools are extremely flexible, but typically involve relatively
steep learning curves. We hope that his article will provide stroke
researchers with a useful introduction to the possibilities offered by
these tools.

The two examples we present are a choropleth and a service catchment
basin estimation. A choropleth is a map display in which regions are
coloured by a measure of the region. Choropleths are the workhorse of
geographical visualization. We use demographic and boundary data from
the Australian Bureau of Statistics and incidence data from the NEMISIS
(2008) {[}g{]} study to estimate stroke cases per postcode and display
the result on an interactive map. The service catchment basin estimation
involves a Monte-Carlo simulation of patients attending a rehabilitation
service of 3 hospitals. The catchment basin of each hospital is the
region that has lower travel time to that hospital than any other.
Catchment basins can be combined with incidence data to estimate load on
rehabilitation centres. The data can be used to explore scenarios, such
as the removal or addition of service centres.

\hypertarget{background}{%
\section{Background}\label{background}}

\hypertarget{geospatial-frameworks}{%
\subsection{Geospatial frameworks}\label{geospatial-frameworks}}

The fundamental unit of geospatial data is a point location. In
practice, most forms of analysis relevant to this discussion will
involve two-dimensional locations, typically represented as a
latitude/longitude pair. More complex data, such as national boundaries
or administrative boundaries consist of sets of points connected
together in defined orders, typically to produce a closed shape. Other
structures, such as road networks, are also constructed using sets of
points and include other types of information, such as speed limits,
travel direction etc. A geospatial framework provides mechanisms for
representing, loading, and saving geospatial data and performing
fundamental mathematical operations. For example, the simple features
(sf) (Pebesma 2018) {[}h{]} package, on which our R examples are based,
provides structures to represent all manner of shapes and associate them
with non spatial quantities, perform transforms between coordinate
systems, display shapes, compute geometric quantities like areas and
distances and perform operations like intersections and unions. The
equivalent python framework is the geopandas package that provides a
geospatial extension to standard dataframes.

\hypertarget{sources-of-regional-data}{%
\subsection{Sources of regional data}\label{sources-of-regional-data}}

The examples below use postcode boundary data available from the
Australian Bureau of Statistics. It is common for boundaries used in
reporting of regional statistics to be available in standard file
formats from the reporting bodies or central authorities along with the
reported statistics. The regional demographics measures, often derived
from national census data, also represent an important source of
information for researchers, including age, sex, income, ethnicity etc.
For example, in the US, key data sources on sociodemographics and the
built environment include the Census Bureau's decennial census (US
Census Bureau n.d.) {[}i{]} (a complete enumeration at fine spatial
scales but coarse, decadal temporal scales), American Community Survey
(Bureau n.d.) {[}j{]} (a survey with annual temporal scales, but often
fairly large standard errors at small spatial scales due to the sample
size), and TIGER/Line shapefiles (Geography n.d.) {[}k{]} of tract,
municipal, and urbanized area boundaries. Additional regional data are
frequently available from municipal, state, county, or metropolitan
governmental agencies.

Demographic data for countries in the European Union are provided by
Eurostat (Eurostat n.d.). This includes time series data from several
years to decades on economics, demography, infrastructure, health,
traffic, and more of the EU {[}Lahti et al. (2017)). Geographic data for
the EU is available through the Geographic Information System of the
COmmission (GISCO), part of Eurostat. This provides

Similar levels of geographic and demographic data are available from
France through INSEE (\url{https://www.insee.fr/en/statistiques}),
Germany through Destatis
(\url{https://www.destatis.de/EN/Homepage.html}), Switzerland through
(\url{https://www.bfs.admin.ch/bfs/en/home.html})

\hypertarget{geocoding-and-reverse-geocoding}{%
\subsection{Geocoding and reverse
geocoding}\label{geocoding-and-reverse-geocoding}}

Location information, such as a patient home address, is often available
as a street address, rather than a coordinate (a latitude/longitude
pair). However operations, such as plotting addresses on a map, require
a coordinate. Geocoding is the process of converting an address to a
coordinate. Reverse geocoding converts a coordinate to an address. A
coordinate is useful in many other types of computation, as we shall see
in the examples below.

There are two common approaches to geocoding and reverse geocoding. The
most ubiquitous is via web services such as Google Maps. Other services,
such as OpenStreetMap's nominatim web service, opencage
(\url{https://opencagedata.com/}), provide similar capabilities and all
can be queried in a automated way from R and python (Salmon 2018). The
other approach is via a local database of geocoded addresses. One
example, for Australia, is the PSMA (formerly Public Sector Mapping
Agencies) address database available in an R queryable form. A local
database allows many high speed queries, but is often less flexible in
terms of query structure than the web services. Web services are
discussed in more detail below.

\hypertarget{distance-and-travel-estimation}{%
\subsection{Distance and travel
estimation}\label{distance-and-travel-estimation}}

A key part of a number of studies cited above is the estimation of
travel time between patient and treatment center. The popularity of
personal navigation systems in smartphones has driven the development of
extremely sophisticated tools to estimate the fastest route between
points. One of the best known, Google Maps (footnote - the two APIs
involvfed are the directions api and distance api), uses a combination
of information about the road network, historic travel time data derived
from smartphone users and live information from smartphones. The travel
time estimates are thus sensitive to time of day, weather conditions and
possibly traffic accidents. Google, and other web services for travel
time estimation, can be queried in a similar fashion to the geocoding
services. It is also possible to create a local database to represent
the road network, allowing more rapid querying, but losing some of the
benefits of traffic models.

\hypertarget{visualization}{%
\subsection{Visualization}\label{visualization}}

Two forms of visualization are used in the following examples - static
and interactive. Static maps are required for printed reports and
typically present a carefully selected view. Interactive maps allow
exploration of a data set, via zooming and toggling of overlays.
Interactive maps often use web services to provide the background map
``tiles'', over which data is superimposed. Different interactive web
services specialise in different types of display. Some tools produce
static and interactive displays in very similar ways.

\hypertarget{introduction-to-web-services}{%
\subsection{Introduction to web
services}\label{introduction-to-web-services}}

Web services providing various forms of geospatial capabilities are a
crucial component of the geospatial analysis tools now available to
researchers. Web services deliver what used to be complex and
specialised information products to the general public. Geocoding and
travel time estimation two common examples that have already been
discussed. Other capabilities include delivery of tiled maps (such as
the Google Map display), street network and building footprint data
(such as from OpenStreetMap), and census data on sociodemographic or
built environment characteristics (such as from the US Census Bureau's
web site).

\hypertarget{application-programming-interfaces-api}{%
\subsubsection{Application programming interfaces
(API)}\label{application-programming-interfaces-api}}

Web services are accompanied via an API. The API allows software tools,
such as R or python, to make requests to the web service and retrieve
results. Thus, if we consider the Google Map example, not only can a
user access a map query for an address via a web browser, but a program
can submit the same request. Furthermore, a program can submit a series
of automated requests. For example, given a list of addresses, it is
relatively simple to generate an R or python procedure to geocode all of
them via a web service.

The ability to use APIs in automated methods also leaves them open to
abuse. In addition, many APIs are commercial products and thus charge
for use, although the use is often free for small volumes.

The combination of these factors tends to mean that many APIs require
somewhat complex setup, typically via signup and creation of keys. Terms
of use may evolve over time, with charging being introduced, possibly
leading to a need to enter credit card details.

We have endeavoured to create examples that do not require keys,
simplifying getting started. However, some extensions have been included
that do require keys. These are described in supplementary material.

\hypertarget{openstreetmap-osm}{%
\subsubsection{OpenStreetMap (OSM)}\label{openstreetmap-osm}}

OSM (\url{https://www.openstreetmap.org/}) is a service collecting and
distributing crowdsourced geospatial data. Many useful OSM services are
available without API keys, and it is thus the platform of choice for
examples in this paper. OSM is also unusual in that allows access to
geospatial structures, such as road networks, rather than images
generated from those structures. This capability is used to estimate
travel time.

\hypertarget{access-to-the-examples}{%
\subsubsection{Access to the examples}\label{access-to-the-examples}}

The examples are available in their source code form from (github).
``Live'' versions are available at (githubpages) and can be viewed in
conjunction with the methods section. The description focuses on the R
versions of the examples. Code is visible in the shaded boxes, while
output of the code, such as maps, are displayed immediately after the
code. Python versions are provided and implement equivalent steps.
Details on downloading and running the examples are available in
supplementary material and at the web site.

\hypertarget{methods}{%
\section{Methods}\label{methods}}

\hypertarget{example-1-choropleth-to-visualize-estimated-stroke-numbers}{%
\subsection{Example 1: Choropleth to visualize estimated stroke
numbers}\label{example-1-choropleth-to-visualize-estimated-stroke-numbers}}

\hypertarget{overview}{%
\subsubsection{Overview:}\label{overview}}

We demonstrate accessing and using different data sources. The first is
Australian Bureau of Statistics census data provided at the postcode
level for population information, stratified by age, as well as postcode
boundary information. The second data source is incidence data from the
North East Melbourne Stroke Incidence Study (NEMESIS). This is combined
with the first dataset to estimate per-postcode stroke incidence. We
demonstrate geocoding by finding the location of a hospital delivering
acute stroke services, and then display postcodes within 15km, colouring
each postcode by estimated stroke incidence.

The steps involved are:

\begin{enumerate}
\def\labelenumi{\arabic{enumi}.}
\item
  Loading census and boundary data: Data from the 2016 Australian
  National Census is available from the Australian Bureau of Statistics
  (\url{https://datapacks.censusdata.abs.gov.au/datapacks/}) , and
  copies are included with examples. The two parts of the data are the
  national postcode boundaries (loaded with the sf::read\_sf command)
  and the demographics, by postcode, for the state of Victoria, loaded
  with the readr::read\_csv command.
\item
  Geocoding hospital location: The coordinates of the hospital of
  interest, Monash Medical Centre, are determined by geocoding the
  hospital address, using the tmaptools::geocode\_OSM command. This
  command using OpenStreetMap to provide the geocoding service.
\item
  Combine demographics and spatial data: An important feature of the
  simple features and geopandas frameworks is the ability to combine
  spatial data, such as postcode boundaries, with associated statistical
  summaries (stroke count, demographics etc). This step uses the
  right\_join function to attach the demographic data to the set of
  postcodes. The right\_join performs two tasks - attaching the
  demographics data and discarding the postcodes for which we don't have
  demographics data (i.e.~those from other states of Australia).
\item
  Compute per-postcode stroke incidence: A column representing stroke
  incidence per postcode is added to the demographics table. The
  computation uses incidence data published by the NEMISIS (Thrift et
  al. 2000) {[}l{]} study to provide rates per 100000 for various age
  ranges. The demographics data also includes population by age range,
  allowing computation of stroke incidence as a weighted sum of
  population columns. Names such as Age\_55\_64\_yr\_P refer to the name
  of a column in the demographics table.
\item
  Compute distance from postcode to hospital: We create a column
  containing the distance from each postcode to the hospital of interest
  using the sf::st\_distance function, which automatically accounts for
  complexities, such as the curvature of the earth. We also set the
  units of quantities to km. We then use the distance in a simple,
  static, choropleth to verify the operation. Cool colours,
  corresponding to small distances are in the expected location.
\item
  Discard remote postcodes: Postcodes further than 20km from the
  hospital are discarded by filtering the data based on the newly
  calculated distance column.
\item
  Interactive display of the result: Finally, an interactive map is
  created using the tmap package. The postcode boundaries are coloured
  according to our estimated stroke count and overlaid on a zoomable map
  provided by OpenStreetMap. Any column in our dataset can be visualized
  in a similar way. A number of useful interactive features are
  available in this style of display, including popup displaying the
  postcode when hovering the mouse over a region and more detailed
  information available when clicking on a region.
\end{enumerate}

\hypertarget{example-2-service-regions-for-stroke-rehabilitation}{%
\subsubsection{Example 2: Service regions for stroke
rehabilitation}\label{example-2-service-regions-for-stroke-rehabilitation}}

In the second example we demonstrate the idea of estimating catchment
basins for a set of three service centres. The idea can be easily
extended to more service centres. A catchment basin, or catchment area
for a service centre is the region that is closer to that service centre
than any other. The definition of ``closer'' is critical in this
calculation, with travel time through the road network being a useful
measure for many practical purposes. The approach used in this example
involves the sampling of random addresses within a region of interest
around the service centres, estimation of travel time from each address
to each service centre, assignment of addresses to the closest service
centre, combination of addresses based on service centre to form
catchment areas. The catchment areas can then be used to estimate
loadings on service centres.

The first four steps, 1) Loading census and boundary data, 2) Geocoding
service centre location and 3) Combining demographics and spatial data
are the same as the previous example, with addresses of multiple service
centres being geocoded. The additional steps are:

\begin{enumerate}
\def\labelenumi{\arabic{enumi}.}
\item
  Compute distance to each service centre from each postcode: A study
  area is generated by computing the distance to each service centre
  frome each postcode and retaining only postcodes within 10km of a
  service centre
\item
  Sample postcodes: A set of random addresses is created for each
  postcode by randomly sampling a database of addresses. The number of
  addresses sampled depends on the sampling approach and the subsequent
  computations, but if local methods are used it is feasible to use
  large numbers of samples. In this case we use 1000 per postcode. Lower
  numbers would be appropriate if subsequent computations required
  charged web services.
\item
  Display sample addresses and postcodes: Display the samples in a map
  form to verify that the distribution matches expected population
  distribution - i.e that there are lower densities in rural areas, and
  that the study area is appropriate.
\item
  Create a street network database: In this example we are employing a
  local approach to travel time estimation. The first step is to fetch a
  road network database from OpenStreetmap and convert to a network form
  for analysis. There are a number of tricks discussed in the online
  document that reduce the size of the download by exploiting knowledge
  of the study area.
\item
  Estimation of travel time: travel time from each address to each
  service centre is then computed using the dodgr::dodgr\_dists
  function, which is optimised to rapidly compute large sets of pairwise
  distances.
\item
  Address-based catchment basins: Each address is assigned to a service
  centre by identifying the centre with the shortest travel time. A view
  using a scatter plot of points coloured by destination is then created
  to verify the result.
\item
  Polygon catchment basins: We convert the pointwise classification to a
  polygon representation using a Voronoi tessellation approach. The
  Voronoi tessellation of a set of points is a set of polygon catchment
  basins, one basin for each point. However the definition of ``closer''
  for the Voronoi basins is based on Euclidean distance rather than road
  network distance or travel time. The Voronoi polygons are from
  addresses assigned to the same service centre are then merged to
  create the polygon representation of the service centre catchment,
  which can be displayed.
\item
  Estimate caseload per centre: The catchment areas can be used in
  conjunction with the per postcode demographics to make estimates. We
  use our per postcode stroke estimate procedure from the previous
  example as a basis for determining the number of rehabilitation cases
  (a simplification for illustration purposes). The sampled addresses
  are the basis for this computation, with the proportion of sampled
  addresses from a postcode assigned to a service centre corresponding
  to the proportion of cases from that postcode attending the centre.
\end{enumerate}

\hypertarget{discussion}{%
\section{Discussion}\label{discussion}}

These examples illustrate fundamental geospatial computational
components in R and python. This includes geocoding with databases and
web services, interactive and static visualization. It also includes
geometric computation of areas and distances, and geospatial
computations of travel time.

The examples in this article illustrate the use of a range of components
that underpin geospatial analysis. By providing an accessible
introduction to these areas, clinicians and researchers can create code
to answer clinically relevant questions on a topics such as service
delivery and service demand. Importantly, these factor in key features
of transport and travel time.

\hypertarget{supplementary-material}{%
\section{Supplementary Material}\label{supplementary-material}}

\hypertarget{api-keys}{%
\subsection{API Keys}\label{api-keys}}

Online services which offer an interface to their applications will
sometimes require use of an API key, or application programming
interface key. This key should be unique for each user, developer or
application making use of the service as it is a way for the provider to
monitor and, where applicable, charge for use.

Two major mapping platforms that require an API key are Google Maps and
Mapbox. At the time of writing both allow unrestricted use of the
mapping API. However, Google has limits on the other services it offers
such as geocoding and direction services.

Setting up API Keys for examples

{[}a{]}citation:
\url{http://stroke.ahajournals.org/content/early/2017/03/29/STROKEAHA.116.015323}
{[}b{]}citation: \url{https://doi.org/10.1161/STROKEAHA.116.015321}
{[}c{]}\url{doi:10.1161/STROKEAHA.114.006293} {[}d{]}citation:
\url{http://stroke.ahajournals.org/content/early/2017/03/29/STROKEAHA.116.015323}
{[}e{]}Is this referring to the Google Distance \emph{Matrix} API?
{[}f{]}cite R and Python. {[}g{]}doi: 10.1111/j.1747-4949.2008.00204.x.
{[}h{]}cite {[}i{]}citation:
\url{https://www.census.gov/history/www/programs/demographic/decennial_census.html}
{[}j{]}citation: \url{https://www.census.gov/programs-surveys/acs/}
{[}k{]}citation:
\url{https://www.census.gov/geo/maps-data/data/tiger-line.html}
{[}l{]}citation:

\begin{verbatim}
@article{thrift2000stroke,
title={Stroke Incidence on the East Coast of Australia The North East Melbourne Stroke Incidence Study (NEMESIS)},
author={Thrift, Amanda G and Dewey, Helen M and Macdonell, Richard AL and McNeil, John J and Donnan, Geoffrey A},
 journal={Stroke},
volume={31},
number={9},
 pages={2087--2092},
year={2000},
publisher={Am Heart Assoc}}
\end{verbatim}

\hypertarget{refs}{}
\leavevmode\hypertarget{ref-Adeoye_2014}{}%
Adeoye, Opeolu, Karen C. Albright, Brendan G. Carr, Catherine Wolff,
Micheal T. Mullen, Todd Abruzzo, Andrew Ringer, Pooja Khatri, Charles
Branas, and Dawn Kleindorfer. 2014. ``Geographic Access to Acute Stroke
Care in the United States.'' \emph{Stroke} 45 (10). Ovid Technologies
(Wolters Kluwer Health): 3019--24.
\url{https://doi.org/10.1161/strokeaha.114.006293}.

\leavevmode\hypertarget{ref-us_census_bureau_acs}{}%
Bureau, US Census. n.d. ``American Community Survey(Acs).'' Accessed
February 19, 2019. \url{https://www.census.gov/programs-surveys/acs}.

\leavevmode\hypertarget{ref-eurostat}{}%
Eurostat. n.d. ``GISCO a Geographic Information System of the
Commission.'' Accessed February 10, 2019.
\url{https://doi.org/10.2785/58916}.

\leavevmode\hypertarget{ref-us_census_tiger_line}{}%
Geography, US Census Bureau. n.d. ``Tiger/Line® Shapefiles and
Tiger/Line® Files.'' Accessed February 19, 2019.
\url{https://www.census.gov/geo/maps-data/data/tiger-line.html}.

\leavevmode\hypertarget{ref-Lahti2017}{}%
Lahti, Leo, Janne Huovari, Markus Kainu, and Przemysław Biecek. 2017.
``Retrieval and Analysis of Eurostat Open Data with the Eurostat
Package.'' \emph{The R Journal} 9 (1): 385--92.

\leavevmode\hypertarget{ref-Milne_2017}{}%
Milne, Matthew S.W., Jessalyn K. Holodinsky, Michael D. Hill, Anders
Nygren, Chao Qiu, Mayank Goyal, and Noreen Kamal. 2017. ``Drip `N Ship
Versus Mothership for Endovascular Treatment.'' \emph{Stroke} 48 (3).
Ovid Technologies (Wolters Kluwer Health): 791--94.
\url{https://doi.org/10.1161/strokeaha.116.015321}.

\leavevmode\hypertarget{ref-Pebesma_2018}{}%
Pebesma, Edzer. 2018. ``Simple Features for R: Standardized Support for
Spatial Vector Data.'' \emph{The R Journal}.
\url{https://journal.r-project.org/archive/2018/RJ-2018-009/index.html}.

\leavevmode\hypertarget{ref-Phan_2017}{}%
Phan, Thanh G., Richard Beare, Jian Chen, Benjamin Clissold, John Ly,
Shaloo Singhal, Henry Ma, and Velandai Srikanth. 2017. ``Googling
Service Boundaries for Endovascular Clot Retrieval Hub Hospitals in a
Metropolitan Setting.'' \emph{Stroke} 48 (5). Ovid Technologies (Wolters
Kluwer Health): 1353--61.
\url{https://doi.org/10.1161/strokeaha.116.015323}.

\leavevmode\hypertarget{ref-R_Core_Team_2018}{}%
R Core Team. 2018. \emph{R: A Language and Environment for Statistical
Computing}. Vienna, Austria: R Foundation for Statistical Computing.
\url{https://www.R-project.org/}.

\leavevmode\hypertarget{ref-opencage}{}%
Salmon, Maëlle. 2018. \emph{Opencage: Interface to the Opencage Api}.
\url{https://CRAN.R-project.org/package=opencage}.

\leavevmode\hypertarget{ref-sanner1999python}{}%
Sanner, Michel F, and others. 1999. ``Python: A Programming Language for
Software Integration and Development.'' \emph{J Mol Graph Model} 17 (1):
57--61.

\leavevmode\hypertarget{ref-thrift_stroke_2000}{}%
Thrift, A. G., H. M. Dewey, R. A. Macdonell, J. J. McNeil, and G. A.
Donnan. 2000. ``Stroke Incidence on the East Coast of Australia: The
North East Melbourne Stroke Incidence Study(Nemesis).'' \emph{Stroke} 31
(9): 2087--92.

\leavevmode\hypertarget{ref-us_census_bureau_decennial}{}%
US Census Bureau, Census History Staff. n.d. ``Decennial Census -
History - U. S. Census Bureau.'' Accessed February 19, 2019.
\url{https://www.census.gov/history/www/programs/demographic/decennial_census.html}.

\leavevmode\hypertarget{ref-2008}{}%
2008. SAGE Publications.
\url{https://doi.org/10.1111/ijs.2008.3.issue-s1}.


\end{document}
